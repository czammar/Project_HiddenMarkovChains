\documentclass[xcolor=dvipsnames, compress]{beamer}
%\usetheme{Madrid} % My favorite!
%\usetheme{Boadilla} % Pretty neat, soft color.
%\usetheme{Warsaw}
%\usecolortheme{dove} 
\usetheme[secheader]{Boadilla}
% \useoutertheme[subsection=false]{smoothbars}
% \useinnertheme{rectangles}
% \usetheme{Marburg}
   \usecolortheme[RGB={139,10,80}]{structure}
%  \usecolortheme[RGB={25,25,112}]{structure}
%\usecolortheme[RGB={255,127,36}]{structure}
% \usetheme{CambridgeUS}
%\usetheme{PaloAlto}
% \usefonttheme{professionalfonts}
% \usepackage{listings}

\usetheme{Boadilla}

% \usetheme{Warsaw}
%\usetheme{Darmstadt} %OK!
%  \usetheme{Frankfurt} %OK!
% \usetheme{Goettingen}
% \usetheme{Dresden}
%\usetheme{JuanLesPins} %OK!!
%\usetheme{Marburg}
%  \usetheme{Montpellier}
% \usetheme{Rochester} %sobrio
%\usetheme{Singapore}
%\usetheme{Szeged}
%\usetheme{Luebeck}

%\usetheme{Hannover}
%\usecolortheme{wolverine}

%\usecolortheme{albatross}
%\usecolortheme{seahorse}
%\usecolortheme{beetle}
%\usecolortheme{crane}
%\usecolortheme{dolphin}
%\usecolortheme{dove} %<- este con orchid
%\usecolortheme{fly}
%\usecolortheme{lily}
%\usecolortheme{orchid}
%\usecolortheme{rose}
%\usecolortheme{seagull}
%\usecolortheme{whale}

%\usetheme{Bergen} % This template has nagivation on the left
%\usetheme{Frankfurt} % Similar to the default 
%with an extra region at the top.
%\usecolortheme{seahorse} % Simple and clean template
%\usetheme{Darmstadt} % not so good
% Uncomment the following line if you want %
% page numbers and using Warsaw theme%
% \setbeamertemplate{footline}[page number]
%\setbeamercovered{transparent}
%\setbeamercovered{invisible}
% To remove the navigation symbols from 
% the bottom of slides%
%\setbeamertemplate{navigation symbols}{} 
%
\usepackage{graphicx}
\usepackage{amssymb,amsmath,amscd}
\usepackage{latexsym,xspace}
\usepackage[utf8]{inputenc}
\usepackage{epsfig}
%\usepackage{fancyhdr}
\usepackage[spanish]{babel}
\usepackage[all]{xy}
\usepackage{enumerate}
\usepackage{eucal}
%\usepackage[usenames]{color}

%#########################
\newcommand{\tx}{\ensuremath{\tau(X)}}
\newcommand{\txx}{\ensuremath{\tau_{X}}}
\newcommand{\Q}{\ensuremath{\mathbb{Q}}}
\newcommand{\Z}{\ensuremath{\mathbb{Z}}}
\newcommand{\N}{\ensuremath{\mathbb{N}}}
\newcommand{\R}{\ensuremath{\mathbb{R}}}
\newcommand{\C}{\ensuremath{\mathbb{C}}}
\newcommand{\A}{\ensuremath{\forall}}
\newcommand{\E}{\ensuremath{\exists}}
\newcommand{\iso}{\ensuremath{\cong}}
\newcommand{\union}{\ensuremath{\cup}}
%\newcommand{\morinyec}{\ensuremath{\precapprox}}
%\newenvironment{prueba}{\vspace{-3mm}\noindent\textbf{Demostraci\'on}\\}{\noindent$\blacksquare$\\}
\newcommand{\nin}{\ensuremath{\notin}}
\renewcommand{\emptyset}{\varnothing}

%\newcommand{\niso}{\ensuremath{\not \cong}}
\newtheorem{teor}{Teorema}[section]
\newtheorem{defi}{Definición}[section]
\newtheorem{ejemplo}{Ejemplos}[section]
\newtheorem{obs}{Observación}[section]
\newtheorem{prop}{Proposición}[section]
\newtheorem{cor}{Corolario}[section]
\newtheorem{ntc}{Notación}[section]
\newtheorem{lema}{Lema}[section]
\newtheorem{prob}{Problema}
\newtheorem{comen}{Comentario}

%\usepackage{bm}         % For typesetting bold math (not \mathbold)
%\logo{\includegraphics[height=0.6cm]{yourlogo.eps}}
%
\title[Hidden Markov Chains]{Hidden Markov Chains}
\author{Names of autors}
\institute[ITAM]
%\date{November 7, 2019}
% \today will show current date. 
% Alternatively, you can specify a date.
%


\begin{document}
%
\begin{frame}
\titlepage
\end{frame}

\begin{frame}
\frametitle{Índice}
 \tableofcontents%[sections]
\end{frame}

\begin{frame}
\section{Hola soy la seccion 1}

Soy un texto de ejemplo

\end{frame}

\begin{frame}

Soy un texto de ejemplo $$\pi(x|y) = \frac{\pi(y|x) \pi(x)}{\pi(y)} $$


\end{frame}

 \begin{frame}
% \section{Motivación}
Ejemplo del ambiente de definiciones:
  \begin{defi}[Hidden Markov Chain Model]
A Hidden Markov Chain Models is ...
 \end{defi}
 \end{frame}



\begin{frame}
\section{Hola soy la seccion 2}
\frametitle{Example of a Theorem}

Ejemplo del ambiente de teoremas:

\begin{theorem}
The quick brown fox jumps over the lazy dog.
\end{theorem}
\end{frame}

%
%\begin{frame}[fragile] % Notice the [fragile] option %
%\frametitle{Verbatim}
%\begin{example}[Putting Verbatim]
%\begin{verbatim}
%\begin{frame}
%\frametitle{Outline}
%\begin{block}
%{Why Beamer?}
%Does anybody need an introduction to Beamer?
%I don't think so.
%\end{block}
% Extra carriage return causes problem with verbatim %
%\end{frame}\end{verbatim} 
%\end{example}
%\end{frame}
 
%\begin{frame}[fragile]  % notice the fragile option, since the body
			% contains a verbatim command
%Example of the \verb|\cite| command to give a reference is below:
%Example of citation using \cite{key1} follows on.
%\end{frame}
 
% \begin{frame}
% \section{Bibliografía}
% \frametitle{Referencias}
% \footnotesize{
% \begin{thebibliography}{99}
%  \bibitem[Morita, 2010]{key1} J. Nagata, K. Morita (1989)
%  \newblock Topics On General Topology.
%  \newblock \emph{Elsevier Science Publisher B.V.} 15(6), 203 -- 243.

% \bibitem[VanMill, 2010]{key1} J. Van Mill ; M. Husek (1992)
%  \newblock Recent Progress In General Topology.
%  \newblock \emph{Elsevier Publications} p. 375.

% \bibitem[MacLane, 2010]{key1} J. S. Mac Lane(1971)
%  \newblock Categories for the working mathematician,.
%  \newblock \emph{Springer} p. 375.

%  \bibitem[Ishii, 2010]{key1} Tadashi Ishii (1969)
%  \newblock On Tychonoff Functor and $w$-Compactness.
%  \newblock \emph{Topology Appl.} 11, 175 -- 187.

%  \bibitem[Ishii, 2010]{key1} T. Hoshina; K. Morita (1980)
%  \newblock On Regular Products Of Topological Spaces.
%  \newblock \emph{Topology Appl.} 11, 47 -- 57.
% \end{thebibliography}
% }
% \end{frame}


 
% \begin{frame}
% %\section{Bibliografía}
% \frametitle{Referencias}
% \footnotesize{
% \begin{thebibliography}{99}
%  \bibitem[Porter, 2010]{key1}J. R. Porter ; R. Grant Woods  (1987)
%  \newblock Extensions and Absolutes of Hausdorff Spaces.
%  \newblock \emph{Springer-Verlag} 856.


%  \bibitem[Simon, 2010]{key1} Petr Simon (1984)
%  \newblock Completely regular modification and products.
%  \newblock \emph{Commentationes Mathematicae Universitatis Carolinae} 25(1), 121--128.



%  \bibitem[Puppier, 2010]{key1} René Puppier (1969)
%  \newblock La Completion Universelle D'un Produit D'espaces Completement Reguliers .
%  \newblock \emph{Publ. Dept. Math, Lyon} 254, 342--351.



% \end{thebibliography}
% }
% \end{frame}
% 
% End of slides
\end{document} 